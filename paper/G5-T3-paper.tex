% This file should be compiled with V2.5 of "sig-alternate.cls" May 2012

\documentclass{sig-alternate}

\begin{document}
%
% --- Author Metadata here ---
% \conferenceinfo{WOODSTOCK}{'97 El Paso, Texas USA}
%\CopyrightYear{2007} % Allows default copyright year (20XX) to be over-ridden - IF NEED BE.
%\crdata{0-12345-67-8/90/01}  % Allows default copyright data (0-89791-88-6/97/05) to be over-ridden - IF NEED BE.
% --- End of Author Metadata ---

\title{State of the art in Anonymity Networks}
\subtitle{Advanced Internet Computing, WS2014}

\numberofauthors{4} 
\author{
\alignauthor
Edvin Kuric\\
       \affaddr{1028327}\\
       \email{e1028327@student.\\
       tuwien.ac.at}
\alignauthor
Philipp Schindler\\
       \affaddr{1128993}\\
       \email{e1128993@student.\\
       tuwien.ac.at}
\and
\alignauthor Richard Bayerle\\
       \affaddr{1025259}\\
       \email{e1025259@student.\\
       tuwien.ac.at}
\alignauthor Julian Kowanz\\
       \affaddr{0727397}\\
       \email{e0727397@student.\\
       tuwien.ac.at}
}
\date{11 January 2015}
\maketitle


\begin{abstract}
This paper provides an overview of the scientific state of the art in anonymity networks.
\end{abstract}

\keywords{Anonymity Networks, Onion Routing, Tor}

\section{Introduction}
TODO

\section{Onion routing}
\emph{Onion routing} is a technique that uses cryptography to provide \emph{anonymous connections}
through the network. It uses \emph{application proxies}, \emph{onion proxies} and
\emph{onion routers} which use regular TCP/IP connections to communicate among themselves
to form virtual \emph{anonymous connections} \cite{reed1998}.

\subsection{Establishing anonymous connections}
As a fist step, the application connects to the \emph{application proxy} the same way it would to
any ordinary proxy server. The \emph{application proxy} must understand the protocol used by the
application (e.g. HTTP) to be able to convert the packets it receives from the application into an
application-independent format. It then connects to the \emph{onion proxy} to request the
creation of a new anonymous connection to the destination the the application intended to connect to.

The \emph{onion proxy} has a list of all \emph{onion routers} it can potentially use to form
the anonymous connection. It selects a chain of routers and creates a special message (the \emph{onion})
that is passed along the chain until it reaches the last router (the \emph{exit funnel}), which then
opens a connection to the recipient originally intended by the application.

The \emph{onion} consists of multiple layers, each intended for one of the \emph{routers} used.
These layers each contain the information about the next \emph{router} in the chain and are encrypted
in such a way that each layer can only be decrypted by the corresponding \emph{router}. So, when receiving
the \emph{onion}, each \emph{router} decrypts (\emph{peels away}) its layer and passes the rest of the
\emph{onion} along to the next \emph{router}. The innermost layer contains information about the
recipient which the \emph{exit funnel} can use to connect to.

Each time a layer is removed, the \emph{onion} shrinks in size. This effect could be used to help traffic
analysis determine the link between originator and recipient. To avoid this issue, a padding of random data
is added each time a layer is removed, so all the \emph{onion} packets look the same from the
outside \cite{ren2009}.

After the \emph{onion} is passed through the chain of \emph{routers}, the \emph{anonymous connection}
is established and can be used to transmit data back and forth.

\subsection{Using anonymous connections}
After the \emph{anonymous connection} has been successfully established, everything sent by the application
to the \emph{application proxy} is simply passed along to the \emph{onion proxy} which then encrypts the
data in layers similar to the way the \emph{onion} was created. The innermost layer, again, is decrypted
by the \emph{exit funnel} and contains the data the application sent.

When data is sent in the other direction (from the recipient back to the application) the encryption layers
are created in reverse order. It will eventually arrive back at the \emph{application proxy}, which can send
it back to the application in plain text thus allowing for two-way anonymous communications.

This whole process is completely transparent to the application because it only ever directly communicates
with the \emph{application proxy}. Applications, therefore, don't need to explicitly support
\emph{Onion routing} in order to use it - all that is needed is an appropriate \emph{application proxy}.

\subsection{Anonymity}
\emph{Onion routing} is designed to protect against traffic analysis by hiding the source and destination of a
packet. Though not its primary purpose, it can also protect against eavesdropping as a side effect because
the communication between \emph{onion routers} is encrypted \cite{reed1998}.

The anonymity provided by \emph{onion routing} comes from the fact that the recipient has no way
of determining the identity of the client it actually communicates with. It cannot differentiate
between direct connections and \emph{anonymous connections} opened by the \emph{exit funnel}.

Assuming the \emph{onion proxy} is trusted, the identities of the originator and the recipient are
never both in clear text at any point. In order to link these two together for traffic analysis,
\emph{all} of the \emph{routers} in the chain would have to be compromised, if at least one
\emph{router} is honest, the anonymity holds \cite{reed1998}.


\section{Tor}
\emph{Tor}, also known as \emph{the second-generation Onion Routing system}, is an
overhaul of the original \emph{onion routing} system that provides several benefits over its predecessor.
These benefits include \emph{perfect forward secrecy}, removing the need for \emph{application proxies},
multiple TCP streams over one \emph{circuit}, \emph{directory servers} and more \cite{tor2004}.

It is free
to use by anyone and its infrastructure is entirely comprised of internet-connected machines that are
set up by volunteers to act as \emph{directory nodes} and \emph{routers}.

\subsection{Terminology}
The virtual connections used by \emph{Tor} to communicate anonymously are referred to as \emph{circuits}
in the literature.

\emph{Entry funnels} and \emph{exit funnels} are called \emph{entry nodes} and \emph{exit nodes},
respectively.

\subsection{Changes from first-generation onion routing}
This is a non-comprehensive list of some of the improvements \emph{Tor} made over 
\emph{first-generation onion routing}. For a complete list, see \cite{tor2004}.

\subsubsection{Perfect forward secrecy}
\emph{Tor} provides \emph{perfect forward secrecy} by using \emph{session keys} for encryption.
If an attacker were to compromise a node, it could not decrypt old traffic it has routed before
because the corresponding session key would have been deleted by then.

\subsubsection{No application proxy}
\emph{Tor} eliminates the need for an \emph{application proxy}. Instead, the (locally running) 
\emph{onion proxy} uses the standard SOCKS interface to communicate with applications.

This simplifies the design and makes the network usable with more types of applications.

\subsubsection{Directory nodes}
In \emph{first-generation onion routing}, \emph{routers} sent out status messages to neighbouring
\emph{routers}, which propagated them further to spread throughout the network. \emph{Tor} uses
a more centralised approach by introducing \emph{directory nodes}.

\emph{Directory nodes} are - as the name implies - directories where users can retrieve a list of
routers that can be used to form a \emph{circuit}. Directory nodes need to be trustworthy because
a malicious directory node could cause the user to only use routers under its control, rendering
the anonymity of the user ineffective. To circumvent this problem, \emph{Tor} only uses a few,
well-known directory nodes that are synchronised so they all send the same list of routers. \cite{tor2004}

\subsubsection{Reusing circuits}
In \emph{first-generation onion routing}, the whole process of creating an \emph{onion} and sending
it through the \emph{routers} to create a \emph{circuit} was done every time the user's application
opened a new connection. This caused a lot of overhead in applications that use short-lived connections.

\emph{Tor}, instead, routes all TCP traffic through one \emph{circuit} for as long as it is active.
\emph{Circuits} do have a timeout though, so new \emph{circuits} (with a different chain of \emph{routers})
are periodically generated to make traffic analysis harder.

\subsection{Vulnerabilities}

\subsubsection{Denial of service}
Tor \emph{routers} are relatively vulnerable to denial of service attacks because a malicious
\emph{onion proxy} can force the \emph{router} to do expensive cryptographic operations while
the \emph{proxy} requires few computational resources to carry out the attack \cite{tor2004}.

\emph{Tor} is also particularly vulnerable to a \emph{router} that temporarily stops to function
because it breaks all \emph{circuits} that go through that router. As opposed to normal network
traffic, where packets are then just routed differently, there is no fail safe for this in \emph{tor}
and the application loses the connection.

\subsubsection{End-to-end timing correlation}
An attacker who observes patterns of traffic into- and out of the \emph{Tor} network can link
the two involved parties of a connection together with high priority based on the timing correlation
of the connections to the entry- and exit nodes \cite{tor2004}.

\subsubsection{Website fingerprinting}
An attacker could load a website to observe patterns in the traffic that such a request creates.
These observed patterns can then be saved as the website's \emph{fingerprint}. If the attacker
can then observe the traffic between a \emph{Tor} user and his \emph{entry node}, he could watch
for this \emph{fingerprint} and infer that the user loaded the same website with high probability.

Of course this attack can be scaled up so the attacker has an entire database of \emph{website
fingerprints}, which could allow him to get a relatively complete picture of the user's browsing
habits even if no direct connection to anything other than the \emph{entry node} is observed
\cite{panchenko2011}.

\subsubsection{Replay attack}
If an attacker has control of both the \emph{entry-} and \emph{exit nodes} of a circuit, he
can carry out a \emph{replay attack}\cite{pries2007} to break the user's anonymity.

To carry out this attack, the \emph{entry node} duplicates one of the cells used to carry data
across the \emph{circuit} which causes an decryption error to occur on the \emph{exit node}.
Since the timing of the duplication and the error correlates, the attacker can confirm that
these two nodes are used together in the same \emph{circuit}. The attacker can then simply
look up the client's identity in the \emph{entry node} and the server's identity in the
\emph{exit node}.

\section{Mixnet}
TODO


\section{Crowds}
TODO

\section{Tarzan}
TODO


\bibliographystyle{abbrv}
\bibliography{G5-T3-paper}
% You must have a proper ".bib" file
%  and remember to run:
% latex bibtex latex latex
% to resolve all references
%

\end{document}
