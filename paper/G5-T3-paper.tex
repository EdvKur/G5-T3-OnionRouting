% This file should be compiled with V2.5 of "sig-alternate.cls" May 2012

\documentclass{sig-alternate}

\begin{document}
%
% --- Author Metadata here ---
% \conferenceinfo{WOODSTOCK}{'97 El Paso, Texas USA}
%\CopyrightYear{2007} % Allows default copyright year (20XX) to be over-ridden - IF NEED BE.
%\crdata{0-12345-67-8/90/01}  % Allows default copyright data (0-89791-88-6/97/05) to be over-ridden - IF NEED BE.
% --- End of Author Metadata ---

\title{State of the Art in Anonymity Networks}
\subtitle{Advanced Internet Computing, WS2014}

\numberofauthors{4} 
\author{
\alignauthor
Edvin Kuric\\
       \affaddr{1028327}\\
       \email{e1028327@student.\\
       tuwien.ac.at}
\alignauthor
Philipp Schindler\\
       \affaddr{1128993}\\
       \email{e1128993@student.\\
       tuwien.ac.at}
\and
\alignauthor Richard Bayerle\\
       \affaddr{1025259}\\
       \email{e1025259@student.\\
       tuwien.ac.at}
\alignauthor Julian Kowanz\\
       \affaddr{0727397}\\
       \email{e0727397@student.\\
       tuwien.ac.at}
}
\date{11 January 2015}
\maketitle


\begin{abstract}
This paper provides an overview of the scientific state of the art in anonymity networks.
\end{abstract}

\keywords{Anonymity Networks, Onion Routing, Tor}

\section{Introduction}
While connections can be encrypted to hide the contents of the transmission, for example using TLS and SSL, the source and destination are still clearly visible.
\\
This is where anonymity networks come in.
The service at the destination does not know where the request originated from, as it only sees the most recent endpoint.
Some even go as far as to hide even the service from the user, using intermediary nodes to communicate.
\\
\\
Unfortunately, this is not just a theoretical concept that is fun to think about.
\\
In some parts of the world anonymity is just the luxury to not be constantly tracked on the web by websites, or on a whole different scale, by law enforcement.
The latter often happens for no particular reason through mass storage of connection data.
\\
In other parts though, anonymity can help access withheld information, or save a person releasing such information from persecution.
\\
All in all, it is not surprising that \emph{TOR} was first developed by the U.S. Naval Research Laboratory, which shows that it is an important topic.
It still is important today and will stay current for the foreseeable future, as new research is done on flaws and improvements\cite{craven2010}.

The topics analysed in this paper are two of the most popular low-latency networks, \emph{Tor} and \emph{I2P}.

\section{Onion routing}
\emph{Onion routing} is a technique that uses cryptography to provide \emph{anonymous connections}
through the network. It uses \emph{application proxies}, \emph{onion proxies} and
\emph{onion routers} which use regular TCP/IP connections to communicate among themselves
to form virtual \emph{anonymous connections} \cite{reed1998}.

\subsection{Establishing anonymous connections}
As a fist step, the application connects to the \emph{application proxy} the same way it would to
any ordinary proxy server. The \emph{application proxy} must understand the protocol used by the
application (e.g. HTTP) to be able to convert the packets it receives from the application into an
application-independent format. It then connects to the \emph{onion proxy} to request the
creation of a new anonymous connection to the destination the the application intended to connect to.

The \emph{onion proxy} has a list of all \emph{onion routers} it can potentially use to form
the anonymous connection. It selects a chain of routers and creates a special message (the \emph{onion})
that is passed along the chain until it reaches the last router (the \emph{exit funnel}), which then
opens a connection to the recipient originally intended by the application.

The \emph{onion} consists of multiple layers, each intended for one of the \emph{routers} used.
These layers each contain the information about the next \emph{router} in the chain and are encrypted
in such a way that each layer can only be decrypted by the corresponding \emph{router}. So, when receiving
the \emph{onion}, each \emph{router} decrypts (\emph{peels away}) its layer and passes the rest of the
\emph{onion} along to the next \emph{router}. The innermost layer contains information about the
recipient which the \emph{exit funnel} can use to connect to.

Each time a layer is removed, the \emph{onion} shrinks in size. This effect could be used to help traffic
analysis determine the link between originator and recipient. To avoid this issue, a padding of random data
is added each time a layer is removed, so all the \emph{onion} packets look the same from the
outside \cite{ren2009}.

After the \emph{onion} is passed through the chain of \emph{routers}, the \emph{anonymous connection}
is established and can be used to transmit data back and forth.

\subsection{Using anonymous connections}
After the \emph{anonymous connection} has been successfully established, everything sent by the application
to the \emph{application proxy} is simply passed along to the \emph{onion proxy} which then encrypts the
data in layers similar to the way the \emph{onion} was created. The innermost layer, again, is decrypted
by the \emph{exit funnel} and contains the data the application sent.

When data is sent in the other direction (from the recipient back to the application) the encryption layers
are created in reverse order. It will eventually arrive back at the \emph{application proxy}, which can send
it back to the application in plain text thus allowing for two-way anonymous communications.

This whole process is completely transparent to the application because it only ever directly communicates
with the \emph{application proxy}. Applications, therefore, don't need to explicitly support
\emph{Onion routing} in order to use it - all that is needed is an appropriate \emph{application proxy}.

\subsection{Anonymity}
\emph{Onion routing} is designed to protect against traffic analysis by hiding the source and destination of a
packet. Though not its primary purpose, it can also protect against eavesdropping as a side effect because
the communication between \emph{onion routers} is encrypted \cite{reed1998}.

The anonymity provided by \emph{onion routing} comes from the fact that the recipient has no way
of determining the identity of the client it actually communicates with. It cannot differentiate
between direct connections and \emph{anonymous connections} opened by the \emph{exit funnel}.

Assuming the \emph{onion proxy} is trusted, the identities of the originator and the recipient are
never both in clear text at any point. In order to link these two together for traffic analysis,
\emph{all} of the \emph{routers} in the chain would have to be compromised, if at least one
\emph{router} is honest, the anonymity holds \cite{reed1998}.


\section{Tor}
\emph{Tor}, also known as \emph{the second-generation Onion Routing system}, is an
overhaul of the original \emph{onion routing} system that provides several benefits over its predecessor.
These benefits include \emph{perfect forward secrecy}, removing the need for \emph{application proxies},
multiple TCP streams over one \emph{circuit}, \emph{directory servers} and more \cite{tor2004}.

It is free
to use by anyone and its infrastructure is entirely comprised of internet-connected machines that are
set up by volunteers to act as \emph{directory nodes} and \emph{routers}.

\subsection{Terminology}
The virtual connections used by \emph{Tor} to communicate anonymously are referred to as \emph{circuits}
in the literature.

\emph{Entry funnels} and \emph{exit funnels} are called \emph{entry nodes} and \emph{exit nodes},
respectively.

\subsection{Changes from first-generation onion routing}
This is a non-comprehensive list of some of the improvements \emph{Tor} made over 
\emph{first-generation onion routing}. For a complete list, see \cite{tor2004}.

\subsubsection{Perfect forward secrecy}
\emph{Tor} provides \emph{perfect forward secrecy} by using \emph{session keys} for encryption.
If an attacker were to compromise a node, it could not decrypt old traffic it has routed before
because the corresponding session key would have been deleted by then.

\subsubsection{No application proxy}
\emph{Tor} eliminates the need for an \emph{application proxy}. Instead, the (locally running) 
\emph{onion proxy} uses the standard SOCKS interface to communicate with applications.

This simplifies the design and makes the network usable with more types of applications.

\subsubsection{Directory nodes}
In \emph{first-generation onion routing}, \emph{routers} sent out status messages to neighbouring
\emph{routers}, which propagated them further to spread throughout the network. \emph{Tor} uses
a more centralised approach by introducing \emph{directory nodes}.

\emph{Directory nodes} are --- as the name implies --- directories where users can retrieve a list of
routers that can be used to form a \emph{circuit}. Directory nodes need to be trustworthy because
a malicious directory node could cause the user to only use routers under its control, rendering
the anonymity of the user ineffective. To circumvent this problem, \emph{Tor} only uses a few,
well-known directory nodes that are synchronised so they all send the same list of routers. \cite{tor2004}

\subsubsection{Reusing circuits}
In \emph{first-generation onion routing}, the whole process of creating an \emph{onion} and sending
it through the \emph{routers} to create a \emph{circuit} was done every time the user's application
opened a new connection. This caused a lot of overhead in applications that use short-lived connections.

\emph{Tor}, instead, routes all TCP traffic through one \emph{circuit} for as long as it is active.
\emph{Circuits} do have a timeout though, so new \emph{circuits} (with a different chain of \emph{routers})
are periodically generated to make traffic analysis harder.

\subsubsection{Hidden services}
\emph{Tor} allows the creation of so-called \emph{hidden services} which are then only reachable from inside
the \emph{Tor} network.
They are not only called hidden because they cannot be reached from the regular web, but also because they
are only accessible via a pseudonym instead of their IP, so that their location is actually hidden.
This protects the provider of a service from prosecution, as opposed to only the user being anonymous\cite{biryukov2013}.
\\
One of the goals of this feature was application--transparency, so assuming correct configuration of the
\emph{Tor} software, the server does not need to be changed and is often a regular web
server\cite{tor2004}.
Other types of services, e.g. chat servers, exist as well\cite{biryukov2013}.
\\
\\
The technique making this possible are \emph{rendezvous points}.
To make a service available as a \emph{Tor hidden service}, the owner has to create a long--term public key pair.
Now some \emph{onion routers} have to be chosen as \emph{introduction points} and advertised to the lookup service.
These \emph{introduction points} have to be signed with the generated public key and can be changed at a later time as well.
Now a \emph{circuit} to each of these introduction points is established, which wait for requests.
\\
A Tor user wanting to access this server has to learn of its pseudonym and ask the lookup service for the introduction points.
The pseudonym is simply the hash of the public key, and ends with \texttt{.onion}.
A random \emph{Tor node} is then chosen as the aforementioned \emph{rendezvous point} and an introduction point asked to relay this information to the \emph{Hidden Service}.
This \emph{rendezvous point} will then be used for communication between the user and the service.
Note that this way:
\begin{itemize}
\item{Both the user and service know the \emph{rendezvous point}, but not each other}
\item{There is still a circuit between each of them and the \emph{rendezvous point}}
\item{The \emph{rendezvous point} does not know anything (neither the location of either end, nor the transmitted information)}
\end{itemize}
\cite{syverson2006}

\subsection{Vulnerabilities}
\subsubsection{Denial of service}
Tor \emph{routers} are relatively vulnerable to denial of service attacks because a malicious
\emph{onion proxy} can force the \emph{router} to do expensive cryptographic operations while
the \emph{proxy} requires few computational resources to carry out the attack \cite{tor2004}.
\\
\emph{Tor} is also particularly vulnerable to a \emph{router} that temporarily stops to function
because it breaks all \emph{circuits} that go through that router. As opposed to normal network
traffic, where packets are then just routed differently, there is no fail safe for this in \emph{tor}
and the application loses the connection.
\\
\emph{Hidden Services} cannot be attacked directly using a DDoS--attack.
The only other way is attacking the whole network.
Unfortunately for the network, this does happen from time to time and leads to not only one site being unusable,
but the whole network\cite{syverson2006}.

\subsubsection{End-to-end timing correlation}
Since no delay is introduced in between hops, an attacker who observes patterns of traffic into- and out of the \emph{Tor} network can link the two involved parties of a connection together with high priority based on the timing correlation of the connections to the entry- and exit nodes \cite{tor2004}.

\subsubsection{Website fingerprinting}
An attacker could load a website to observe patterns in the traffic that such a request creates.
These observed patterns can then be saved as the website's \emph{fingerprint}. If the attacker
can then observe the traffic between a \emph{Tor} user and his \emph{entry node}, he could watch
for this \emph{fingerprint} and infer that the user loaded the same website with high probability.

Of course this attack can be scaled up so the attacker has an entire database of \emph{website
fingerprints}, which could allow him to get a relatively complete picture of the user's browsing
habits even if no direct connection to anything other than the \emph{entry node} is observed
\cite{panchenko2011}.

\subsubsection{Replay attack}
If an attacker has control of both the \emph{entry-} and \emph{exit nodes} of a circuit, he
can carry out a \emph{replay attack}\cite{pries2007} to break the user's anonymity.

To carry out this attack, the \emph{entry node} duplicates one of the cells used to carry data
across the \emph{circuit} which causes an decryption error to occur on the \emph{exit node}.
Since the timing of the duplication and the error correlates, the attacker can confirm that
these two nodes are used together in the same \emph{circuit}. The attacker can then simply
look up the client's identity in the \emph{entry node} and the server's identity in the
\emph{exit node}.

\section{I2P}
I2P is the common abbreviation of \emph{Invisible Internet Project}.
It is based on \emph{Garlic Routing} which will be explained in detail in the next section.
Even though it shares many of its functioning principles with \emph{Tor}, it has a different focus.

\subsection{Garlic Routing Terminology}
As the name suggests, it is inspired by \emph{Onion Routing} which was explained earlier.
This section will compare some basic design decisions.

\subsubsection{Cloves}
\emph{Tor} uses \emph{cells} of fixed size to hide information about the data transmitted between nodes, but a temporal context of the \emph{cell} entering and exiting a circuit can still be established.
Here, so--called \emph{cloves} are used instead.
\\
In essence, they are multiple \emph{cells} packed together, but there are additional refinements.
Each node packs a different amount of \emph{cells} into a \emph{clove}, and random padding is added.
Additionally, there are delay, delivery and priority instructions for each \emph{cell}.
\\
So instead of just receiving and forwarding packets, the node has to decrypt the \emph{clove} to extract the different \emph{cells} and then pack them into different \emph{cloves} depending on the additional instactions and the local \emph{clove} size before encrypting and forwarding them to the next\cite{zantout2011}.
\\
\\
The term is a bit confusing, since a single cell of a garlic is usually called ``clove'', but on the official website it is explained that it is their term for garlic bulbs\cite{garlic}.

\subsubsection{Tunnels}
Although the idea of a \emph{circuit} still exists, the approach described in the previous subsection makes it clear that \emph{I2P} is more of a messaged--based system.
Even \emph{cells} that belong to the same connection can and will take a different route through the network, as each of them is repacked into different \emph{cloves}.
This also implies that the response will use a different path.
\\
\\
Each \emph{I2P} client creates two \emph{inbound} and \emph{outbound} \emph{tunnels} each by selecting available nodes.
They are changed every 10 minutes.
For sending a message, a \emph{clove} is send through the \emph{outbound tunnel}, and as soon as it is at the end of its path, it will continue to hop until it arrived at the destination.
The sender therefore does not know about the path the message will take\cite{olivier2011}.
\subsubsection{NetDB}
As described before, \emph{Tor} uses trusted directory nodes to supply the client with a \emph{circuit}.
This can be seen as a reliability bottleneck, since there are not many of them.
\emph{I2P} uses \emph{Kademlia}, which is an implementation of a \emph{distributed hash table} (DHT), instead.
It is a decentralised lookup table with great performance, scalability and fault tolerance properties.
This part of the system is called \emph{NetDB}.
\\
\\
When a client first connects to the network, it does use one of the predefined ``good'' hosts though.
Then, additional nodes are discovered.

\subsubsection{Services}
\emph{Exit nodes} which are used to access the internet are a central part of the \emph{Tor} network.
In contrast, the base principle of \emph{I2P} is communication between \emph{I2P} clients and the number of out-proxies is limited\cite{olivier2011}.
The focus on applications can be easily seen by looking at the programs coming bundled with the client.
They include a \emph{BitTorrent} client named \emph{I2PSnark}, an email client named \emph{susimail}, and even a web server to host a website\cite{i2p}.
These are called \emph{Eepsites} and use the fake domain \texttt{.i2p}.
\\
\\
According to \cite{olivier2011}, more than 25\% of clients are using the \emph{I2PSnark Bittorrent} client, as opposed to about 3\% of \emph{Tor} users.
Another 30\% of applications could not be tagged.
In any case, this proves a diversity of uses, as was intended by the developers.

\subsection{Vulnerabilities}
\subsubsection{Improvements}
\emph{I2P} might have achieved some of its goals in improving \emph{Onion Routing}, but not all of them, and introduced some new ones.
\\
For example, by using \emph{cloves} instead of \emph{cells}, \emph{end--to--end correlation attacks} were mitigated, as each \emph{cell} inside of a \emph{clove} can have its own delay.
\\
By using a \emph{DHT} for routing information, the network was decentralised and therefore made more robust\cite{zantout2011}.

\subsubsection{Denial of Service}
Unfortunately for the network, there are still other types of DoS attacks to be performed.
\\
For example, a \emph{starvation attack} could be performed, during which the attacker lets a big number of malicious nodes join the network and drop packets after establishing a connection.
Participating with a lot of malicious nodes is considered a classic attack on anonymization networks and is called \emph{sybil attack}.
\\
Considering that the \emph{I2P} network is much smaller than \emph{Tor}, resources can be easily depleted by injecting one node into the network that generates huge amounts of traffic\cite{zantout2011}.

\subsubsection{Deanonymization of users}
In \cite{egger2013}, an attack that combined several steps was performed to obtain the IP of clients with a high probability (about 80\%).
\\
The first step was to take over most of the \emph{netDB} by replacing the active nodes one by one.
As the number is regulated by the network and usually the participation is automatic, a malicious node with activated participation can be started up while DoSing an actual node.
At some point it will drop its participation for performance reasons, and assuming the amount of needed \emph{DHT} nodes did not increase, the node was successfully replaced by a malicious one.
\\
The second step is a classical \emph{Sybil} attack as described before to have many nodes in the topographical vicinity of the victim.
\\
Now there is a high probability that any kind of interaction the victim has with the network will be with one of the nodes, some of which are participating in the \emph{DHT}.
This together with the fact that storing to the \emph{DHT} is done in the clear can be then used to link the target and the observed IP.

\subsubsection{Information disclosure through users}
As mentioned before, each \emph{I2P} client comes with a web server so that the user can run his or her own \emph{Eepsite}.
This is different from \emph{Tor}, as additional effort is required to set up a hidden service.
\\
Since it is so easy, it is not improbable that this feature will be used by inexperienced users.
In fact, servers of \emph{Eepsites} often respond with a verbose header that allows for profiling if the same machine runs an unrelated site on the open web.
\\
\\
A lot of public websites have huge security issues that even allow for code execution, e.g. using the PHP \texttt{shell\_exec} function in an unsafe way.
This can be used to send a request to a machine owned by the attacker, if an \emph{Eepsite} does the same\cite{crenshaw2011}.


\bibliographystyle{abbrv}
\bibliography{G5-T3-paper}
% You must have a proper ".bib" file
%  and remember to run:
% latex bibtex latex latex
% to resolve all references
%

\end{document}
